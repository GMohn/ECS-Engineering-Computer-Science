% 
% file: ECS 32B Homework 1 
% author: Geoffrey Mohn
% student ID: 912568148
% History: 
%   Apr 12, 2019 - created
%   Apr 16, 2019 - last revised 
%
% This is LaTeX template to get you started using LaTeX
% for making problem-set solutions.
%

\documentclass[11pt]{article}
\usepackage{amsmath}
\usepackage{amsfonts}
\usepackage{amssymb}
\usepackage{enumitem}
\usepackage{color}
\usepackage[dvipsnames]{xcolor}
\setlength{\oddsidemargin}{0in}
\setlength{\evensidemargin}{0in}
\setlength{\textheight}{8.5in}
\setlength{\textwidth}{6.0in}
\setlength{\topmargin}{-1in}

% Sample macros -- how you define new commands
% My own set of frequently-used macros have grown to many hundreds of lines.
% Here are some simple samples.

\newcommand{\Adv}{{\mathbf{Adv}}}          %example macro 
\newcommand{\getsr}{{\:\stackrel{{\scriptscriptstyle\hspace{0.2em}\$}}{\leftarrow}\:}}  % a more complex sample macro
\newcommand{\Func}[1]{{\mathrm{Fun}[{#1}]}}       % These macros take one
\newcommand{\Randd}[2]{{\mathrm{Rand}[{#1},{#2}]}} % and two arguments


%%%%%%%%%%%%%%%%%%%%%%%%%%%%%%%%%%%%%%%%%%%%%%%%%%%%%%%%%%%%%%%%%%%%%%%%%%%
\title{\bf ECS 32B Homework 1\\[2ex] 
	\rm\normalsize ECS 32B --- Spring 2019}
\date{\today}
\author{\bf Geoffrey Mohn\ ID: 912568148}

\begin{document}
	\maketitle
	
	
	%%%%%%%%%%%%%%%%%%%%%%%%%%%%%%%%%%%%%%%%%%%%%%%%%%%%%%%%
	\section*{Problem 1} 
	{\underline{Complexity of a Code Sample}}\\*\\
	{\bf For this problem you will analyze the complexity of the following bit of instructive Python code that does not solve anything important.}
\[ \begin{array}{l}
\mbox {\color{red}a = 4}\\
\mbox {\color{red}b = 10}\\
\mbox {\color{green}for\thickspace  i\thickspace  in\thickspace  range(n):}\\
\mbox  {\color{green} \indent \indent for\thickspace j\thickspace in\thickspace range(a):}\\        
\mbox	\indent \indent \indent {\color{blue}total = total + 1}\\
\mbox \indent {\color{purple}for\thickspace i\thickspace in\thickspace range(b):}\\
\mbox  \indent \indent  {\color{BurntOrange}total = total + 1}\\
\mbox {\color{Aquamarine}print(total)}
\end{array}\]
\begin{itemize}
\item[a)] {\bf State a function T(n) for this code in terms of n. The function T(n) should give the number of statements executed by the Python interpreter as a function of the integer variable n. Explain your reasoning.}
\\
\begin{align*}
T(n) &=  {\color{red}O(1)}+{\color{red}O(1)}+{\color{green}O(1)+O(1)}+{\color{blue}(n^{2}\times O(1))}+{\color{purple}O(1)}+{\color{BurntOrange}(n\times O(1))}+{\color{Aquamarine}O(1)}
\end{align*}
\underline{Reasoning}: {\color{red}The first 2 O(1) initializes the variables ‘a’ and ‘b’ which is ran in a constant amount of time.} {\color{green}The 3rd and 4 O(1) sets up  the for loop of i in range of n, and the for loop of j in range of a respectively which are also ran in a constant amount of time.} {\color{blue} The initialized variable 'total' (O(1)) in the nested for loops is iterated $n^2$ times as the for loops are executed n(for i in range(n)) $\times$ n (for j in range(a)) times. } {\color{purple}The following O(1) sets up the for loop for i in range of b}. {\color{BurntOrange}The variable 'total' is intialized and ran n times for i in range(b).}  {\color{Aquamarine}The print statement takes a constant amount of time and is expressed as O(1) as well.} \\
{\bf T(n) = 2+2+$n^2$+1+n+1}\\
{\bf or T(n) = $n^2$ + n + 6}\\

\item[b)] {\bf Give the smallest worst-case (big-O) complexity for this code in terms of n that works. Formally prove it by finding c and $n_{0}$ such that T(n) $\leq$ cf(n) for n $\geq$ $n_{0}$}\\\

T(n) = {$n^2+  n+6$}\\ 
T(n) $\leq$ cf(n)\\
{$n^2+ n+6$}$\leq$ $cn^2$ \\
{$n^2+ n$}$\leq$ $cn^2-6$\\
{n($n+ 1$)}$\leq$ $cn^2-6$\\
{($n+ 1$)}$\leq$ $cn-\frac{6}{n}$\\
{$n$}$\leq$ $cn-\frac{6}{n}-1$ \quad\quad\quad try $c = 2$, try $n_{0}=6$\\
$6 \leq 12 - 2$\\
$6 \leq 10$ = True\\
{\bf \color{PineGreen} Choosing c = 2 and $n_{0}$ = 6, we have shown that $n^2$ + n + 6 is O($n^2$) because}\\
{\bf \color{PineGreen}$n^2+n+6 \leq 2n^2$ for $n \geq 6$}

\section*{Problem 2} 
	{\underline{Exponential Complexity}}\\*\\
	{\bf Suppose the number of operations required by a particular algorithm is exactly T(n) = $2^n$ and our 1.6 Ghz computer performs exactly 1.6 billino operations per second. What is the largest problem, in terms of n, that can be solved in under a second? In under a day?}\\*\\
T(n) = $2^n$ \thickspace1.6 Ghz computer performing 1,600,000,000 operations per second\\
1,600,000,000 = $2^n$\\
$log_{2}$(1,600,000,000) = $log_{2}$($2$)$\times n$\\
$log_{2}$(1,600,000,000) = n\\
$n \approx 30.5$\\
86,400 seconds in a day\\
$log_{2}$(1,600,000,000)$\times 86,400$\\
$\approx 2,641,716.7$ \\
{\bf \color{PineGreen}O($2^n$) where the largest problem in terms of n, is $n \approx 30.5$ per second and $\approx 2,641,716.7$ per day} \pagebreak


\section*{Problem 3} 
	{\underline{The Traveling Salesman Problem}}\\*\\
	{\bf Given a list of cities and the distances in between them, the task is to find the shortest possible tour that starts at a city, vists each city exactly once and returns to a starting city.A particular tour can be described as list of all cities [c1, c2, c3, ..., cn] ordered by the position in which they are visited with the assumption that you return from the last city to the start.}\\*\\
n = number of cities\\
m = n $\times$ n matrix of distances\\
min = $\infty$\\
the for loop checks randomly and takes all possible outcomes then checks each outcome for shortest distance, this would check $n!$ times.\\
suppose $n = 4 \therefore m = 16$ $\forall$ possible tours theres 4 starting locations, once one is picked there are 3 remaining destinations to go after, then 2 and after that 1.\\ the for loops would iterate $4 \times 3 \times 2 \times 1$ or, 4!\\
{\bf \color{PineGreen}the (big-O) complexity would be n! or O(n!)}\\

\section*{Problem 4} 
	{\underline{Complexity Bound Types}}\\*\\
	{\bf Formal proofs are not required here but briefly explain your reasoning.}\\*\\
{\bf is $log_{2}(n) O(n)$?}\\
yes, $log_{2}(n)$ is logarithmic, whereas O(n) is linear. O(n)denotes an upper bound. O(n) is above $log_{2}(n)$ so O(n) shows the upper bound and can therefore be lower than O(n)\\*\\
{\bf is $log_{2}(n) \Omega(n)$?}\\
no, $log_{2}(n)$ is below $\Omega(n)$ and $\Omega(n)$ denotes a lower bound. Therefore anything below the lower bound cannot be $\Omega(n)$\\*\\
{\bf is $log_{2}(n) \Theta(n)$?}\\
no, $\Theta$ denotes an asymptotically tight upper and lower bound. if the function is $\Theta(n)$ then it is also O(n) and $\Omega$(n). $\Omega$(n) is not $log_{2}(n)$ because $log_{2}(n)$ is lower than the lower bound $\Omega(n)$.	

\pagebreak

\section*{Problem 5} 
	{\underline{Calculating Bounds}}\\*\\
	{\bf Suppose an algorithm solves a problem of size n in at most T(n) = $2n^3 + n^2 + 1$ steps.}
\item[a)] {\bf Prove that T(n) is O ($n^3)$. Show your work including values for c and $n_{0}$}\\
T(n) = $2n^3 + n^2 + 1$\\
$2n^3 + n^2 + 1 \leq cn^3$\\
$2n^3 + n^2 \leq cn^3 - 1$\\
$n^2(2n + 1) \leq cn^3 - 1$\\
$2n + 1 \leq cn - \frac{1}{n^2}$\\
$2n \leq cn - \frac{1}{n^2} - 1$\thickspace \thickspace {\color{blue} let $n_{0}$ = 1 and c = 4}\\
$2(1) \leq (4)(1) - \frac{1}{1} -1$\\
$2 \leq 2$
\item[b)] {\bf Prove that T(n) is $\Theta(n^3)$ by proving that it is also $\Omega(n^3)$ Show your work including values for c and $n_{0}$ }\\
show T(n) is $\Omega(n)$\\
T(n) = $2n^3 + n^2 + 1$\\
$2n^3 + n^2 + 1 \geq cn^3$\\
$2n^3 + n^2 \geq cn^3$ - 1\\
$n^2(2n + 1) \geq cn^3$ - 1\\
$(2n + 1) \geq cn - \frac{1}{n^2}$\\
$2n \geq cn - \frac{1}{n^2} - 1$\thickspace {\color{blue} let $n_{0}$ = 1 and c = 2}\\
$2(1) \geq (2)(1) - \frac{1}{1} - 1$\\
$2 \geq 0$\\
so T(n) is also $\Omega(n) \therefore$ T(n) is $\Theta(n)$ since T(n) is $\Omega(n$) and T(n) is $\Omega(n)$\\

\end{itemize}
\end{document}


Discussion
big o is the upper bound worst case computational performance
big omega is the lower bound in the worst case you cannot improve performance over this bound
t(n) -> o(n) 
T(n) <= c(n)
c > o
n > n_0
theta(n) its the upper and lower bound 
c_1n lower bound big omega <= T(n) <= c_2n upper bound big o
cf(n) will always be the worst case
T(n) = 3n +10 (10 statements and a loop with 3 assignments in the loop)
T(n) = 3n - 10 prove T(n) >= cn


