% 
% file: ECS 20 Homework 1 
% author: Geoffrey Mohn
% student ID: 912568148
% History: 
%   Oct 4, 2016 - created
%   Oct 5, 2016 - last revised 
%
% This is LaTeX template to get you started using LaTeX
% for making problem-set solutions.
%

\documentclass[11pt]{article}
\usepackage{amsmath}
\usepackage{amsfonts}
\usepackage{amssymb}
\usepackage{enumitem}
\setlength{\oddsidemargin}{0in}
\setlength{\evensidemargin}{0in}
\setlength{\textheight}{9in}
\setlength{\textwidth}{6.5in}
\setlength{\topmargin}{-0.5in}

% Sample macros -- how you define new commands
% My own set of frequently-used macros have grown to many hundreds of lines.
% Here are some simple samples.

\newcommand{\Adv}{{\mathbf{Adv}}}          %example macro 
\newcommand{\getsr}{{\:\stackrel{{\scriptscriptstyle\hspace{0.2em}\$}}{\leftarrow}\:}}  % a more complex sample macro
\newcommand{\Func}[1]{{\mathrm{Fun}[{#1}]}}       % These macros take one
\newcommand{\Randd}[2]{{\mathrm{Rand}[{#1},{#2}]}} % and two arguments


%%%%%%%%%%%%%%%%%%%%%%%%%%%%%%%%%%%%%%%%%%%%%%%%%%%%%%%%%%%%%%%%%%%%%%%%%%%
\title{\bf ECS 20 Homework 2\\[2ex] 
	\rm\normalsize ECS 20 --- Fall 2016}
\date{\today}
\author{\bf Geoffrey Mohn\ ID: 912568148}

\begin{document}
	\maketitle
	
	
	%%%%%%%%%%%%%%%%%%%%%%%%%%%%%%%%%%%%%%%%%%%%%%%%%%%%%%%%
	\section*{Problem 1} 
	{\underline{Construct a truth table for each of these compound propositions:}}\\*\\

	\begin{itemize}
		\item [] a) 
	p $\wedge$ q $\rightarrow$ p $\vee$ q
	
	\begin{center}
		\begin{tabular}{||c | c | c | c | c||} 
			\hline
			p & q & p $\wedge$ q & p $\vee$ q & p $\wedge$ q $\rightarrow$ p $\vee$ q \\ [0.5ex] 
			\hline\hline
			T & T & T & T & T \\ 
			\hline
			T & F & F & T & T\\
			\hline
			F & T & F & T & T\\
			\hline
			F & F & F & F & T\\
			\hline
			
		\end{tabular}
	\end{center}
	
		\item [] b)
		(q $\rightarrow$ $\neg$p) $\leftrightarrow$ (p $\leftrightarrow$ q)
		
	\begin{center}
		\begin{tabular}{||c | c | c | c | c | c||}
			\hline
			p & q & $\neg$p & (q $\rightarrow$ $\neg$p) & (p $\leftrightarrow$ q) & (q $\rightarrow$ $\neg$p) $\leftrightarrow$ (p $\leftrightarrow$ q)\\
			[0.5ex]
			\hline\hline
			T & T & F & F & T & F\\
			\hline
			T & F & F & T & F & F\\
			\hline
			F & T & T & T & F & F\\
			\hline
			F & F & T & T & T & T\\
			\hline
		\end{tabular}
	\end{center}
		\item [] c)
	(p $\leftrightarrow$ q) $\oplus$ (p$\leftrightarrow$ $\neg$q)
	\end{itemize}
		\begin{center}
			\begin{tabular}{||c | c | c | c | c | c||}
				\hline
				p & q & $\neg$q & p $\leftrightarrow$ q & p $\leftrightarrow$ $\neg$q & (p $\leftrightarrow$ q) $\oplus$ (p $\leftrightarrow$ $\neg$q)\\
				[0.5ex]
				\hline\hline
				T & T & F & T & F & T\\
				\hline
				T & F & T & F & T & T\\
				\hline
				F & T & F & F & T & T\\
				\hline
				F & F & T & T & F & T\\
				\hline
			\end{tabular}
		\end{center}
	
	\section*{Problem 2}
	{\underline{Construct a truth table for each of these compound propositions:}}\\*\\
	\begin{itemize}
	\item [] a)
	(p $\oplus$ q) $\vee$ (p $\leftrightarrow$ $\neg$q)\\*\\
		
		\begin{center}
			\begin{tabular}{||c | c | c | c | c | c||}
				\hline
				p & q & $\neg$q & p $\oplus$ q & p $\leftrightarrow$ $\neg$q & (p $\oplus$ q) $\vee$ (p $\leftrightarrow$ $\neg$q)\\
				[0.5ex]
				\hline\hline
				T & T & F & F & F & F\\
				\hline
				T & F & T & T & T & T\\
				\hline
				F & T & F & T & T & T\\
				\hline
				F & F & T & F & F & F\\
				\hline
			\end{tabular}
		\end{center}
	\item [] b)\\*\\
	(p $\oplus$ q) $\wedge$ (p $\oplus$ $\neg$q)
	
		\begin{center}
			\begin{tabular}{||c c c c c c||}					\hline
				p & q & $\neg$q & p $\oplus$ q & p $\oplus$ $\neg$q & (p $\oplus$ q) $\wedge$ (p $\oplus$ $\neg$q)\\
				[0.5ex]
				\hline\hline
				T & T & F & F & T & F\\
				\hline		
				T & F & T & T & F & F\\
				\hline
				F & T & F & T & F & F\\
				\hline
				F & F & T & F & T & F\\
				\hline			
			\end{tabular}
		\end{center}
	

	
	\end{itemize}
	
	%%%%%%%%%%%%%%%%%%%%%%%%%%%%%%%%%%%%%%%%%%%%%%%%%%%%%%%%
	\newpage
	\section*{Problem 3} 
	
	A contestant in a TV game show is presented with three boxes, A, B, and C. He is told
	that one of the boxes contains a prize, while the two others are empty. Each box has a
	statement written on it:
	Box A: The prize is in this box
	Box B: The prize is not in box A
	Box C: The prize is not in this box
	The host of the show tells the contestant that only one of the statements is true. Can the
	contestant find logically which box contains the prize? Justify your answer.\\*\\
	
	Box C.\\*\\
	 If Box A is true and has the prize, then box C's statement is also true which breaks the fact that only one box is true.\\*\\
	 If Box B is true then Box A and C have to be false then the prize is in Box C
	 If Box C is true and doesn't have the prize, then only box A and B have the prize. While Box C is true, Box A and Box B must be false. Which leads to a paradox between A and B.\\*\\
	  Box C is the prize because then Box A doesn't have the prize, Box B is the truth saying that A doesn't contain the prize and Box C is false meaning it does have the prize.
	

			
	
	%%%%%%%%%%%%%%%%%%%%%%%%%%%%%%%%%%%%%%%%%%%%%%%%%%%%%%%%
	\section*{Problem 4} 
	
This exercise relate to the inhabitants of the island of knights and knaves created by
Smullyan, where knights always tell the truth and knaves always lie. You encounter two
people, A and B. Determine, if possible, what A and B are if they address you in the way
described. If you cannot determine what these two people are, can you draw any
conclusions?
A says “The two of us are both knights”, and B says “A is a knave”.
\\*\\
B is a Knight and A is a knave. \\*\\
A can't be a knight because then both A and B have to be  knights, but B is calling A a knave. \\*\\
If B is a knave then A has to be a knight based on B's statement.\\*\\
If B is a knight, then A is a knave and lying making the statements okay.

\newpage
	%%%%%%%%%%%%%%%%%%%%%%%%%%%%%%%%%%%%%%%%%%%%%%%%%%%%%%%%
	\section*{Problem 5} 

Use truth tables to verify the associative laws:
\begin{itemize}
	\item [] a)
	(p $\vee$ q) $\vee$ r $\Leftrightarrow$ p $\vee$ (q $\vee$ r)\\
	
		\begin{center}
			\begin{tabular}{||c | c | c | c | c | c | c||}					\hline
				p & q & r & (p$\vee$q) & (q $\vee$ r) & (p $\vee$ q) $\vee$ r & p$\vee$(q $\vee$ r)\\	
				\hline
				T & T & T & T & T & T & T\\
				\hline
				T & T & F & T & T & T & T\\
				\hline
				T & F & T & T & T & T & T\\
				\hline
				T & F & F & T & F & T & T\\
				\hline			
				F & T & T & T & T & T & T\\
				\hline
				F & T & F & T & T & T & T\\
				\hline	
				F & F & T & F & T & T & T\\
				\hline
				F & F & F & F & F & F & F\\
				\hline
	
				\end{tabular}
			\end{center}
			
	\item [] b)
	(p $\wedge$ q)$\wedge$r $\Leftrightarrow$ p $\wedge$(q $\wedge$ r)\\
	
\end{itemize}
			\begin{center}
				\begin{tabular}{||c | c | c | c | c | c | c||}					\hline
					p & q & r & (p$\wedge$q) & (q $\wedge$ r) & (p $\wedge$ q) $\wedge$ r & p$\wedge$(q $\wedge$ r)\\	
					\hline
					T & T & T & T & T & T & T\\
					\hline
					T & T & F & T & F & F & F\\
					\hline
					T & F & T & F & F & F & F\\
					\hline
					T & F & F & F & F & F & F\\
					\hline			
					F & T & T & F & T & F & F\\
					\hline
					F & T & F & F & F & F & F\\
					\hline	
					F & F & T & F & F & F & F\\
					\hline
					F & F & F & F & F & F & F\\
					\hline
					
				\end{tabular}
			\end{center}
	%%%%%%%%%%%%%%%%%%%%%%%%%%%%%%%%%%%%%%%%%%%%%%%%%%%%%%%%
	\section*{Problem 6}
	Show that p $\leftrightarrow$ q and (p $\wedge$ q)$\vee$($\neg$p $\wedge$ $\neg$q) are equivalent.
	
			\begin{center}
				\begin{tabular}{||c | c | c | c | c | c | c | c||}					\hline
					p & q & $\neg$p & $\neg$q & p $\wedge$ q & ($\neg$p $\wedge$ $\neg$q) & (p $\wedge$ q) $\vee$ ($\neg$p $\wedge$ $\neg$q)& p $\leftrightarrow$ q\\
					[0.5ex]
					\hline\hline
					T & T & F & F & T & F & T & T\\
					\hline		
					T & F & F & T & F & F & F & F\\
					\hline
					F & T & T & F & F & F & F & F\\
					\hline
					F & F & T & T & F & T & T & T\\
					\hline			
				\end{tabular}
			\end{center}
			
			p $\leftrightarrow$ q and (p $\wedge$ q)$\vee$($\neg$p $\wedge$ $\neg$q) have the same truth value and therefore logically equivalent.\\
			\newpage
		
	%%%%%%%%%%%%%%%%%%%%%%%%%%%%%%%%%%%%%%%%%%%%%%%%%%%%%%%%
	\section*{Problem 7}
	Use either a truth table, or logical equivalences, to show the equivalence:\newline
	($\neg$(p $\vee$ F) $\wedge$($\neg$ q $\wedge$ T)) $\vee$ p $\Leftrightarrow$ (q $\rightarrow$ p)
	
		
		\begin{center}
			\begin{tabular}{||c | c | c | c | c | c | c | c | c | c | c ||}					\hline
				p & q & $\neg$p & $\neg$q & T & F & $\neg$(p $\vee$ F) & ($\neg$q $\wedge$ T) & $\neg$(p $\vee$ F) $\wedge$ ($\neg$q $\wedge$ T) & ($\neg$(p $\vee$ F) $\wedge$ ($\neg$q $\wedge$ T)) $\vee$ p & q $\rightarrow$ p\\
				[0.5ex]
				\hline\hline
				T & T & F & F & T & F & F & F & F  & T & T\\
				\hline		
				T & F & F & T & T & F & F & T & F & T & T\\
				\hline
				F & T & T & F & T & F & T & F & F & F & F\\
				\hline
				F & F & T & T & T & F & T & F & T & T & T\\
				\hline			
			\end{tabular}
		\end{center}
	\newline
	%%%%%%%%%%%%%%%%%%%%%%%%%%%%%%%%%%%%%%%%%%%%%%%%%%%%%%%%
	\section*{Problem 8}
	Use two different methods to show the equivalence:
	($\neg$(q $\rightarrow$ p))$\vee$ (p $\wedge$ q)$\Leftrightarrow$q\\*\\
	

	\begin{center}
	LHS = ($\neg$(q $\rightarrow$ p))$\vee$(p$\wedge$q)\\*\\
	 ($\neg$(p$\vee$ $\neg$q)$\vee$(p$\wedge$q) conditional statement\\*\\
	 $\neg$($\neg$p$\wedge$($\neg$q))$\vee$(p$\wedge$q) be morgan's law \\*\\
	 (p $\wedge$ q) $\vee$ (p $\wedge$ q) double negation \\*\\
	 (p$\wedge$q) idempotent \\*\\
	 q specialization\\*\\
	 LHS q $\Leftrightarrow$ q RHS
	\end{center}
	
				\begin{center}
					\begin{tabular}{||c | c | c | c | c | c | c ||}					\hline
						p & q & $\neg$p & $\neg$q & ($\neg$(q $\rightarrow$ p))& (p $\wedge$ q) & ($\neg$(q $\rightarrow$ p))$\vee$(p$\wedge$q)\\
						[0.5ex]
						\hline\hline
						T & T & F & F & F & T & T \\
						\hline		
						T & F & F & T & F & F & F \\
						\hline
						F & T & T & F & T & F & T \\
						\hline
						F & F & T & T & F & F & F \\
						\hline			
					\end{tabular}
				\newline
					($\neg$(q $\rightarrow$ p))$\vee$(p$\wedge$q)has the same truth table values as q and therefore logically equivalent
				\end{center}
	%%%%%%%%%%%%%%%%%%%%%%%%%%%%%%%%%%%%%%%%%%%%%%%%%%%%%%%%
	\section*{Extra Credit}
	We are back on the island of knights and knaves (see exercise 4 above). John and Bill are
	residents.
	John: if Bill is a knave, then I am a knight
	Bill: we are different
	Who is who?
	\newline\newline
	John and Bill are both knaves.\newline John's statement is false because if Bill is a knave, then John is also a knave. Bill's statement is false because if they are both knaves then they are not different. Both of their statements are false and therefore they are both knaves.
\end{document}